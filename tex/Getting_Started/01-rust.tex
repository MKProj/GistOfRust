\chapter{What is Rust?}

\par Rust was born in the research labs of Mozilla, and was aimed to be 
a memory safe, zero abstractions language. Not only is it statically typed (knows the data type at compile time), 
but is a high-level programming langauge able to perform low-level tasks. It is being considered the replacement of 
C/C++ due to it's safety with memory and threads (in concurrent tasks), and this is all thanks to Rust's strict ownership rules. 

\par Now Rust does have a steep learning curve, however if you come from languages such as C++, OCAML, you may have it easier than 
others that come from langauges like Python. However even though it has a steep learning curve, it is definitely worth the investment, and as a fellow Rustacean, it definitely changes your perspective of handling situations. 

\par \noindent Here are a few things that Rust does that I just love: 
\begin{itemize}
    \item Rust has a nice package manager called \textbf{Cargo}
    \item Rust has documentation syntax that is used to well 
    document your \textit{Crate (A Rust Package)} 
    \item Rust's compiler is very helpful and can help solve your problem at times
\end{itemize}

\par \noindent To really know if Rust is for you, it's good to know what it can be used for, so here's some common use cases: 
\begin{itemize}
    \item Systems Programming (Kernel Modules, CLI Applications, etc.)
    \item Web Servers (Rcoket, Actix, Warp)
    \item Desktop Applications (GTK)
\end{itemize}

\par In the next few sections, we will discuss core ideas to learn about Rust, but we still recommend doing some more reading on your own. The Rust Foundation provides their own documentation on various topics, and can be found at [https://www.rust-lang.org/learn](https://www.rust-lang.org/learn).

\par To install, visit [https://rustup.rs](https://rustup.rs) and follow the instructions it gives to your particular system. 